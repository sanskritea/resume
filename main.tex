%----------------------------------------------------------------------------------------
%	PACKAGES AND OTHER DOCUMENT CONFIGURATIONS
%----------------------------------------------------------------------------------------
\documentclass[margin, centered, 11pt]{res}
\usepackage[bottom=0.5in]{geometry}
\usepackage{comment}
\usepackage{xcolor}



\topmargin=-0.40in
\oddsidemargin -.5in
\evensidemargin -.5in
\textwidth=6.5in
\itemsep=0in
\parsep=0in
\linespread{1}
\newsectionwidth{1in}
\usepackage[pdftex]{graphicx}
\usepackage{enumitem}
\usepackage{wrapfig}
\usepackage{helvet}
% \usepackage[utf8]{inputenc}
% \usepackage[T1]{fontenc}
% \usepackage{ebgaramond}

\definecolor{vlpink}{RGB}{244,231,236}
\definecolor{mpink}{RGB}{194,123,160}
\definecolor{lpink}{RGB}{213,166,189}
\definecolor{dpink}{RGB}{166,77,121}
\definecolor{dgrey}{RGB}{67,67,67}

\pagecolor{vlpink}

\usepackage[colorlinks = true,
            linkcolor = dpink,
            urlcolor  = dpink,
            citecolor = dpink,
            anchorcolor = dpink]{hyperref}
            
\setlength{\textwidth}{6.5in} % Text width of the document
\setlength{\textheight}{700pt}
\usepackage{tabto}

\begin{document}

%----------------------------------------------------------------------------------------
%	NAME AND ADDRESS SECTION
%----------------------------------------------------------------------------------------\
\begin{center}
\hspace{-\hoffset}
\textbf{\fontsize{20}{1}\selectfont SANSKRITI CHITRANSH}
\end{center}
\\
\vspace{-12mm}
\begin{center}
    \linespread{1}
    \hspace{-\hoffset} 
    \href{mailto:chitransh.sanskriti@gmail.com}{chitransh.sanskriti@gmail.com}  ~\raisebox{0.3ex}{\scalebox{0.5}{\textbullet}}~ \(+91\) 7007651934 
\end{center}
\vspace{-5mm}
\moveleft\hoffset\vbox{\hrule width 19cm height 0.8pt}
\\
\vspace{-7mm}
\begin{resume}

%----------------------------------------------------------------------------------------
%	EDUCATION SECTION
%----------------------------------------------------------------------------------------

\section{Education}

\textbf{\href{http://www.bits-pilani.ac.in/}{Birla Institute of Technology and Science (BITS), Pilani}}\\
\textbf{M.Sc. (Hons.) Physics} \hfill 2015 - 2019\\
CGPA : 8.59/10


\textbf{City Montessori School, Lucknow} (Council for the Indian School Certificate Examination)\\ High School : 96.75\% \hfill 2013 - 2015 \\
\textbf{{Loreto Convent Intermediate College, Lucknow}} (Council for the Indian School Certificate Examination)\\
Secondary School : 94.00\% \hfill 2007 - 2013

%----------------------------------------------------------------------------------------
%	EXPERIENCE SECTION
%----------------------------------------------------------------------------------------
\section{Research Experience}
\textbf{Research Assistant}
\\
\textbf{\href{https://www.iisc.ac.in/en/}{Indian Institute of Science, India}}\\
\emph{Advisor : \href{https://sites.google.com/view/sqd-lab/home}{Prof. Vibhor Singh} } \hfill January 2020 - present\\
\textbf{Ground State Cooling of Electromechanical Oscillator with a Transmon }
\begin{itemize}[noitemsep, label=\raisebox{0.35ex}{\tiny$\bullet$}]
\item Reviewing opto-mechanical cooling theory to implement ground state cooling of a nano-mechanical resonator coupled to a flux-driven transmon 
\item Simulating cooling schemes with $\sigma_z - \sigma_x$ and $\sigma_x - \sigma_x$ resonator-qubit coupling to optimise qubit parameters for effective ground state cooling of the oscillator from bath thermal temperature $\sim 150$ phonons.
 \end{itemize}

\textbf{Research Assistant}
\\
\textbf{\href{http://www.tifr.res.in/}{Tata Institute of Fundamental Research, India}}\\
\emph{Advisor : \href{http://www.tifr.res.in/~quantro/vijay/index.htm}{Prof. R. Vijayaraghavan} } \hfill June 2019 - December 2019\\
\textbf{Towards Implementing All-to-All Coupling in Superconducting Qubits}
\begin{itemize}[noitemsep, label=\raisebox{0.35ex}{\tiny$\bullet$}]
\item Reviewed inter-qubit coupling in trapped ion systems and superconducting circuits to implement all-to-all qubit coupling in a small scale superconducting quantum processor.
\item Longitudinally coupled the qubits and the cavity such that qubits remain decoupled and communicate only via the cavity (analogous to inter-ion coupling in trapped ion chains through the ion-chain motional modes)
 \item Simulated first and second order transitions to implement controlled-NOT gate and explored microwave cavity architecture and device fabrication
 \end{itemize}

\textbf{Master's Thesis}
\\
\textbf{\href{https://www.quantumlah.org/}{Centre for Quantum Technologies, Singapore}}\\
\emph{Advisor : \href{https://coldiongroup.wixsite.com/index/manas}{Prof. Manas Mukherjee}} \hfill June 2018 - November 2018\\
\textbf{A Novel Ion Source for Quantum Computing: A Prototype Development}
\begin{itemize}[noitemsep, label=\raisebox{0.35ex}{\tiny$\bullet$}]
\item Built an ion source for Ion Trap Quantum Computing based on an original design to overcome defects like trap instability and anomalous heating faced by common ion sources (resistive ovens, laser ablation of targets)
\item Conducted extensive survey of trap loading techniques, trap defects and devised alternate trap loading scheme
\item Designed required components for the loading scheme and assembled ultra-high vacuum, optical and imaging systems to test the ion source
 \end{itemize}

% \textbf{Summer Research Intern}
% \\
% \textbf{\href{http://www.iucaa.ernet.in/}{The Inter-University Centre for Astronomy and Astrophysics (IUCAA), Pune, India}}
% \\
% \emph{Mentored by \href{http://www.iucaa.ernet.in/~anand/}{Prof. Raghunathan Srianand}} \hfill May 2016 - July 2016\\
% \textbf{Analysis of Quasar Absorption Lines from SDSS Photometric Data} - Using photometric data of quasars with absorbers in their line of sight taken from the Sloan Digital Sky Survey (SDSS), we used some image processing techniques such as stacking to establish a correspondence between the results already obtained from the spectral data also taken from SDSS. We used some statistical methods to establish this result. \\
% \\
% \textbf{Summer Research Intern}\\
% \textbf{\href{http://www.ncra.tifr.res.in/}{The National Centre for Radio Astrophysics (NCRA-TIFR), Pune, India}}\\
% \emph{Mentored by \href{http://www.ncra.tifr.res.in/ncra/people/academic/ncra-faculty/Yashwant_Gupta}{Prof. Yashwant Gupta}} \hfill May 2015 - July 2015\\
% \textbf{Testing and Debugging the Transient Detection Pipeline of GMRT} - Squashed crucial bugs and tested the transient pipeline using test data from known and reliable transient sources such as pulsars. Also reviewed key concepts of radio astronomy and pulsar astrophysics in the process.


%----------------------------------------------------------------------------------------
%	Selected Projects Section
%----------------------------------------------------------------------------------------
\section{Selected Projects}

\textbf{Superconducting Circuits and Applications in Quantum Information Devices}\\
\emph{Advisor : \href{https://www.bits-pilani.ac.in/Pilani/jayendra/Profile}{Prof. Jayendra N Bandyopadhyay}, BITS Pilani} \hfill January 2018 - May 2018
\begin{itemize}[noitemsep, label=\raisebox{0.35ex}{\tiny$\bullet$}]
\item Comprehensively studied superconducting devices from the basics of solid-state theory
\item Covered essential elements of superconducting devices (Josephson Junctions, SQIDs, flux, phase and charge qubits, Transmons) and explored their use in hybrid quantum circuits, quantum gates and quantum state preparation
\end{itemize}

\textbf{Experimental Techniques in Quantum Optics}\\
\emph{Advisor : \href{https://www.bits-pilani.ac.in/Pilani/jayendra/Profile}{Prof. Jayendra N Bandyopadhyay}, BITS Pilani} \hfill August 2017 - December 2017
\begin{itemize}[noitemsep, label=\raisebox{0.35ex}{\tiny$\bullet$}]
\item Performed an in-depth study of contemporary experimental techniques in quantum optics
\item Studied the fundamentals of quantum optics theory (classical and quantum models of light, optical instruments, lasers) and experimental applications of linear and non-linear optics (photo-detection, squeezing).
\end{itemize}\\


%----------------------------------------------------------------------------------------
%	RELEVANT COURSE SECTION
%----------------------------------------------------------------------------------------
\section{Relevant \hspace{2mm} Courses}
Quantum Mechanics, Quantum Optics, Quantum Information and Computing, Solid State Physics, Atomic and Molecular Physics, Mathematical Methods in Physics, Statistical Mechanics and Computational Physics

%----------------------------------------------------------------------------------------
%	TECHNICAL SKILLS SECTION
%----------------------------------------------------------------------------------------
\section{Technical \hspace{2mm} Skills}

\textbf{Nanofabrication} : Electron beam lithography, electron beam evaporation, plasma ashing \\
\textbf{Optical Assembly} : Optical alignment and application, pulsed laser set-up and operation, basics of ECDL development and laser locking techniques\\
\textbf{Ultra-High Vacuum Pumping Systems} : UHV protocols, UHV systems design\\
\textbf{Softwares} :  AutoCAD, COMSOL, CPO, Inventor, Mathematica, Microwave Office, SIMION  \\
\textbf{Programming Languages} : Java, Python (QTip, QCAT)


\section{Schools attended} %Talks \textit{\&} Conferences}
 \textbf{\href{https://theory.ncbs.res.in/physlife}{5th Simons Centre for Study of Living Machines-NCBS Monsoon School}} \hfill{June 2017}\\
\textbf{National Centre for Biological Sciences, Bangalore, India}\\
School on applications of physics and engineering in biology wherein topics of study included the structure and function of bio-molecules, the organization of cells, the development of organisms, populations and ecosystems and aspects of evolution 


%---------------------------------------------------------------------------------------
%	PUBLICATION SECTION
%---------------------------------------------------------------------------------------

%\section{Publications}
%\begin{itemize}[leftmargin=*]
%\item Devanshu J, \textbf{Ashish K}, Rakshit S, Sameer S, ``Recommendation Techniques for Adaptive E-learning'', Advances in Computer Science and Information Technology, vol. 2, No. 1, 2015. \href{https://drive.google.com/file/d/0B6A-3_6rwie9bS1OaFdzbW9BZXM/view?usp=sharing}{view here}
%\item \textbf{Ashish Kedia} and Anusha Prakash, "Data Synchronization on Android Clients", International Conference on Communication Software and Networks, June 6-7$^{th}$, 2015, Chengdu, China. \href{http://ieeexplore.ieee.org/xpl/articleDetails.jsp?reload=true&arnumber=7296156}{view here}
%\end{itemize}


%----------------------------------------------------------------------------------------
%	ACHIEVEMENT SECTION
%----------------------------------------------------------------------------------------

%\section{Scores \textit{\&}  Awards}
%\href{https://www.ets.org/gre/subject/about/content/physics}{Subject %GRE$^\textregistered$ (Physics)} : \textbf{950/990}\hfill{September 2018}  \\
%\href{https://www.ets.org/gre}{General GRE$^\textregistered$} : \textbf{329/340} %\hfill{August 2019}\\
%\href{https://www.ets.org/toefl}{Test of English as a Foreign Language (TOEFLiBT$^\textregistered$)} : \textbf{111/120}
%\hfill{August 2019}


\\
%----------------------------------------------------------------------------------------
%	HOBBIES SECTION
%----------------------------------------------------------------------------------------
\section{Extra-curricular Activities}
\textbf{\href{http://www.tifr.res.in/~outreach/outreach/index.html}{Frontiers of Science}}\\
\textbf{Outreach Program, Tata Institute of Fundamental Research}
\hfill{November 2019 }
\begin{itemize}[noitemsep, label=\raisebox{0.35ex}{\tiny$\bullet$}]
\item Volunteer for the Science Popularisation and Public Outreach Committee of the Tata Institute of Fundamental Research for the promotion of science education and research among school children from rural and urban areas
\item Demonstrated and explained evaporative cooling and electronic noise reduction at low temperatures in the context of quantum computing to school children from around the country
\end{itemize}

\begin{comment}
\textbf{Corroboration and Review Committee, BITS Pilani}
\hfill{August 2016 - May 2018 }
\begin{itemize}[noitemsep, label=\raisebox{0.35ex}{\tiny$\bullet$}]
\item One of three members  (from a batch of 1000) of a constitutional and autonomous committee of the Students’ Union, BITS Pilani 
\item Oversaw annual budget allocation (of INR 1,500,000), financial management and auditing of all student run activities including the student organized festivals Oasis and APOGEE
\end{itemize}
\end{comment}

\textbf{President, MATRIX, Literary Society, BITS Pilani}
\hfill{August 2017 - May 2018}
\begin{itemize}[noitemsep, label=\raisebox{0.35ex}{\tiny$\bullet$}]
\item Led a community of 30-35 literature and cinema enthusiasts
\item Conducted directed discussions on philosophy, writing, art and current affairs, organised presentations and introductions of books, movies etc.  
\end{itemize}

\textbf{Substitute Teacher, St. Ann's Convent School, Lucknow}
\hfill{July 2017}
\begin{itemize}[noitemsep, label=\raisebox{0.35ex}{\tiny$\bullet$}]
\item Class teacher for class VIII, responsible for 30 underprivileged students. Conducted daily activities, maintained class records and organized students' participation in school events  
\item Taught chemistry to classes VI-X, covered topics like classification of elements, reactions, periodic table and electrolysis 
\end{itemize}
\vspace{10mm}
\moveleft\hoffset\vbox{\hrule width 19cm height 1pt}




% \section{More}
% Please visit \href{https://adivijaykumar.github.io/academic/}{https://adivijaykumar.github.io/academic/}

\end{resume}
\end{document}

